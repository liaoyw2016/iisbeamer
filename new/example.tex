%%%%%%%%%%%%%%%%%%%%%%%%%%%%%%%%%%%%%%%%%%%%%%%%%%%%%%%%%%%%%%%%%%%%%%
%%%%%                                                                %
%%%%% Document Class                                                 %
%%%%%                                                                %
%%%%%%%%%%%%%%%%%%%%%%%%%%%%%%%%%%%%%%%%%%%%%%%%%%%%%%%%%%%%%%%%%%%%%%
\documentclass{beamer}

%%%%% Package loading
\usepackage[utf8]{inputenc}
\usepackage[english]{babel}
\usepackage[T1]{fontenc}

\usepackage{lmodern}
\usepackage{xcolor}
\usepackage{listings}


%%%%% IIS theme-specific settings.
\usetheme[%
% waferpng %% Use the PNG version of the wafer image on the title slide (increases file size!).
]{iis}

%% (Optional) Scale the wafer image on the title page.
\setwaferwidth{0.5} % In percent of the \linewidth (default: 0.5)
\setwaferleft{0.0}  % In centimeter from the top-left corner of the slide (default: 0.0)
\setwafertop{5.8}   % In centimeter from the top-left corner of the slide (default: 5.8)

%% (Optional) Change the margin of the text on the slides (default =
%% 2em). Note that we currently do not adapt the indentation of the
%% frametitle accordingly.
% \setbeamersize{text margin left=1em, text margin right=1em}

%% You may change the appearance of the navigation symbols (shown at
%% the bottom right of each slide) using any of the following lines.
% \setbeamertemplate{navigation symbols}{} %% Disable navigation symbols
% \setbeamertemplate{navigation symbols}[vertical] %% Vertically organize the navigation symbols.
% \setbeamertemplate{navigation symbols}[only frame symbol] %% Only show frame symbols


%%%%% Settings, specific only for this presentation.
\newcommand\latexcls[1]{\textsf{#1}}
\newcommand\latexsty[1]{\textsf{#1}}

%% A listing style for LaTeX/TeX code (note that the colors being used
%% are defined in the iis beamer template).
\lstdefinestyle{latexcodestyle}{
  basicstyle        = \ttfamily,              %% The basic font style to be used for the code
  keywordstyle      = \color{eth4}\bfseries,  %% The keyword style
  language          = {[LaTeX]TeX},           %% The language of the code
  texcsstyle        = *\color{eth7}\bfseries, %% Style of LaTeX/TeX-specific code
  moretexcs         = {usetheme},             %% Additional LaTeX/TeX-specifc keywords
  morekeywords      = {iis},                  %% Other keywords
}


%%%%%%%%%%%%%%%%%%%%%%%%%%%%%%%%%%%%%%%%%%%%%%%%%%%%%%%%%%%%%%%%%%%%%%
%%%%%                                                                %
%%%%% Document Settings                                              %
%%%%%                                                                %
%%%%%%%%%%%%%%%%%%%%%%%%%%%%%%%%%%%%%%%%%%%%%%%%%%%%%%%%%%%%%%%%%%%%%%

\title{\textbf{\texttt{iis}} -- The \LaTeX{} \latexcls{beamer} Theme of the Integrated Systems Laboratory}
\subtitle{An easy to use template}
\date{Zurich, May 2015}

%% Determine the authors of the work.
\author[%
  Doe %% Used for the footline.
]{%
  \textbf{Michael Muehlberghuber}%% The author giving the presentation.
  %% Any further authors should come in here.
}

\institute[%
  Integrated Systems Laboratory%% The short version of the institute will be used for the footer.
]{Integrated Systems Laboratory}


%%%%%%%%%%%%%%%%%%%%%%%%%%%%%%%%%%%%%%%%%%%%%%%%%%%%%%%%%%%%%%%%%%%%%%
%%%%%                                                                %
%%%%% Document Start                                                 %
%%%%%                                                                %
%%%%%%%%%%%%%%%%%%%%%%%%%%%%%%%%%%%%%%%%%%%%%%%%%%%%%%%%%%%%%%%%%%%%%%
\begin{document}


\begin{frame}[%
  plain, %% We do not want to have headers and footers on the title page.
  noframenumbering, %% Do not consider the title page for frame numbers.
]
\titlepage
\end{frame}

\begin{frame}
\frametitle{Outline}
\tableofcontents
\end{frame}

\section{Introduction}

\begin{frame}[%
  fragile %% Required to use the listings-specific commands.
]
\frametitle{Introduction and Requirements}
%
This presentation serves as a basic introduction to the \LaTeX{}
\latexcls{beamer} theme of the Integrated Systems Laboratory
(IIS). Its main requirements are listed in the follwing: \vfill
%
  \begin{itemize}
  \item An easy to use \latexcls{beamer} theme, which can be activated
    by adding \lstinline[style=latexcodestyle]!\usetheme{iis}! to a
    \latexcls{beamer} document. \vfill
  \item Make use of the corporate design (CD) colors and other design
    elements (e.g., logos) of ETH
    Zurich.\footnote{\url{https://www1.ethz.ch/hk/docs/corporate_design}}
    \vfill
  \item Create a minimalistic header and footer in order to make most of
    the slide space available for the actual content. \vfill
  \item Provide some pre-defined commands, which adhere to the CD of
    ETH Zurich.
  \end{itemize}
\end{frame}

\subsection{Usage}

\begin{frame}[%
  fragile %% Required to use the listings-specific commands.
]
  \frametitle{Usage}
  \vfill
  \begin{block}{\textbf{Usage at the Integrated Systems Laboratory}}<1->
    In order to use this \latexcls{beamer} theme at the IIS, you only
    need to add the following line to the preamble of your document:
    \lstinline[style=latexcodestyle]!\usetheme{iis}! 
  \end{block}
  %
  \begin{block}{\textbf{External usage}}<2->
    %
    In order to use this \latexcls{beamer} theme on your personal
    laptop or another arbitrary workstation, in addition to the above
    line, you need to make sure that the following files are readable
    for your \LaTeX{} distribution (e.g., in the directory where your
    \LaTeX{} source document lies):
    %
    \begin{center}
    \footnotesize
    \begin{minipage}[b]{0.45\linewidth}
      \begin{itemize}\itemsep0pt
      \item \texttt{beamerthemeiis.sty}
      \item \texttt{beamerouterthemeiis.sty}
      \item \texttt{beamerinnerthemeiis.sty}
      \item \texttt{beamercolorthemeiis.sty}
      \item \texttt{beamerfontthemeiis.sty}
      \end{itemize}
    \end{minipage} \hfill
    \begin{minipage}[b]{0.45\linewidth}
      \begin{itemize}\itemsep0pt
      \item \texttt{eth\_logo.pdf}
      \item \texttt{eth\_logo\_neg.pdf}
      \item \texttt{wafer.jpg}
      \item \texttt{wafer.png}
      \end{itemize}
    \end{minipage} \hfill
    \end{center}
  \end{block}

\end{frame}


\section{Examples}
\subsection{Itemize and Enumerate Environments}

\begin{frame}
\frametitle{Examples}
\framesubtitle{Itemize and Enumerate Environments}
  % 
  \vfill
  \begin{itemize}
  \item<1-> First \texttt{itemize} entry
  \item<2-> Second \texttt{itemize} entry
  \item<3-|alert@3> Third \texttt{itemize} entry
    \begin{itemize}
    \item<4-> First \texttt{itemize} subentry
    \item<5-|alert@5> Second \texttt{itemize} subentry
    \end{itemize}
  \item<6-> Fourth \texttt{itemize} entry
  \end{itemize}
  %
  \vfill
  \vfill
  \begin{enumerate}
  \item<7-> First \texttt{enumerate} entry
  \item<8-|alert@8> Second \texttt{enumerate} entry
  \item<9-> Third \texttt{enumerate} entry
    \begin{enumerate}
    \item<10-> First \texttt{enumerate} subentry
    \item<11-> Second \texttt{enumerate} subentry
    \end{enumerate}
  \item<12-> Fourth \texttt{enumerate} entry
  \end{enumerate}
  \vfill
\end{frame}

\subsection{Description Environment}

\begin{frame}
\frametitle{Examples}
\framesubtitle{Description Environment}
  % 
  \vfill
  %
  Per default, the labels of the \texttt{description} environment are
  right aligned, as shown in the two examples below:
  \setbeamersize{description width of=Second:}
  \begin{description}
  \item[First:] First \texttt{description} entry
  \item[Second:] Second \texttt{description} entry
  \end{description}
  \setbeamersize{description width=5cm}
  \begin{description}
  \item[First:] First \texttt{description} entry
  \item[Second:] Second \texttt{description} entry
  \end{description}
  \vfill
  \vfill
  We also provide a command for left-aligned description labels:
  \setbeamertemplate{description item}[align left]
  \setbeamersize{description width of=Second:}
  \begin{description}
  \item[First:] First \texttt{description} entry
  \item[Second:] Second \texttt{description} entry
  \end{description}
  \vfill
\end{frame}

\subsection{Boxes}

\begin{frame}
  \frametitle{Examples}
  \framesubtitle{Boxes}
  \vfill
  \begin{block}{\textbf{The header of a \texttt{block}}}<1->
    The body of a \texttt{block}
  \end{block}
  \vfill
  \begin{exampleblock}{\textbf{The header of an \texttt{exampleblock}}}<2->
    The body of an \texttt{exampleblock}
  \end{exampleblock}
  \vfill
  \begin{alertblock}{\textbf{The header of an \texttt{alertbox}}}<3->
    The body of an \texttt{alertbox}
  \end{alertblock}
  \vfill
  \begin{beamerboxesrounded}[width=\linewidth,shadow=true]{\textbf{The header of a \texttt{beamerboxesrounded}}}
    The body of a \texttt{beamerboxesrounded}
  \end{beamerboxesrounded}
  \vfill
\end{frame}


\begin{frame}
  \frametitle{Examples}
  \framesubtitle{Boxes cont'd}
  \vfill
  \begin{block}<only@1>{\textbf{The header of a \texttt{block}}}
    The body of a \texttt{block}
  \end{block}
    \begin{block}<only@2->{\textbf{The header of another \texttt{block}}}
    The body of another \texttt{block}
  \end{block}
  \vfill
  \begin{alertblock}<visible@3->{\textbf{The header of an \texttt{alertbox}}}
    The body of an \texttt{alertbox}
  \end{alertblock}
  \vfill
  \begin{exampleblock}<visible@2->{\textbf{The header of an \texttt{exampleblock}}}
    The body of an \texttt{exampleblock}
  \end{exampleblock}
  \vfill
\end{frame}


\begin{frame}
  \frametitle{Examples}
  \framesubtitle{Theorems and Proofs}
  \vfill
  \begin{theorem}<1->
    There exists an infinite set.
  \end{theorem}
  \vfill
  \begin{proof}<2->
    This follows from the axiom of infinity.
  \end{proof}
  \vfill
  \begin{example}<3->[Natural Numbers]
    The set of natural numbers is infinite.
  \end{example}
  \vfill
\end{frame}


\section{Customization}

\begin{frame}
  \frametitle{Customize the Titlepage}
  \begin{itemize}
  \item How to change the position of the wafer image.
  \item When should I use the PNG version of the wafer image on the
    title page?
  \end{itemize}
\end{frame}


\section{Backup Slides}

\begin{frame}
  \frametitle{Backup Slides}
  \begin{itemize}
  \item Describe how to use backup slides.
  \end{itemize}
\end{frame}


\section{FAQ}

\begin{frame}
  \frametitle{FAQ}
  \framesubtitle{Basics}
  \begin{itemize}
  \item Where did you get the color composition for this theme from?
  \item Why don't we use the standard \LaTeX{} \latexsty{beamer}
    template provided by
    ETH
    Zurich?\footnote{\url{https://www1.ethz.ch/hk/docs/corporate_design/buero}}
  \end{itemize}
\end{frame}

\begin{frame}
  \frametitle{FAQ}
  \framesubtitle{Collaborations}
  \begin{itemize}
  \item Describe how to create title pages for collaborations.
  \end{itemize}
\end{frame}




%%%%%%%%%%%%%%%%%%%%%%%%%%%%%%%%%%%%%%%%%%%%%%%%%%%%%%%%%%%%%%%%%%%%%%
%%%%%                                                                %
%%%%% Document End                                                   %
%%%%%                                                                %
%%%%%%%%%%%%%%%%%%%%%%%%%%%%%%%%%%%%%%%%%%%%%%%%%%%%%%%%%%%%%%%%%%%%%%
\end{document}

